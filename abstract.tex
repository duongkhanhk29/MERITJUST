This research explores the paradoxes of meritocracy using two distinct investigations. The first, employing Boolean analysis, uncovers an irony in meritocracy: it is perceived as fair, yet it downplays hard work (the very architect of fairness) and favors talent. It proposes three potential conditions (not necessarily simultaneously) for a logical meritocracy: difference legitimation, absence of genetic constraints on capabilities, or the use of effort as the yardstick for merit. The second investigation, utilizing Machine Learning on data from the European Social Survey round 9, underscores that meritocratic equality of opportunities goes hand in hand with the belief in social privilege. This bolsters the viewpoint that belief in meritocracy is associated with social inequality. The research concludes by endorsing the development of a novel form of meritocracy.