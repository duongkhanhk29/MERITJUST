To simplify the interpretation of the Boolean algebraic calculation, I replaced the logical symbols with the following substitutions: \(X \implies Y\) became \(X \to Y\); \(X \cap Y\) became \(X \cdot Y\); \(X \cup Y\) became \(X + Y\), and \(\overline{X}\) is the negation of \(X\). I commence with Scenario 4:
\[
\big((L \to (G + N)) \cdot (G \to J) \cdot (G \to P) \cdot (N \to ((P + L) \cdot O)) \cdot (O \to J) \cdot (P \to \overline{J})\big) \to (L \cdot O \to J)
\]
Applying the equivalence \(X \to Y = \overline{X} + Y\) to each premise and the conclusion then the entire statement, the expression is transformed to:
\[
\big((\overline{L} + G + N) \cdot (\overline{G} + J) \cdot (\overline{G} + P) \cdot (\overline{N} + ((P + L) \cdot O)) \cdot (\overline{O} + J) \cdot (\overline{P} + \overline{J})\big)^{\overline{}} + \big((L \cdot O)^{\overline{}} + J\big)
\]
Further utilizing the equivalence \((X \cdot Y)^{\overline{}} = \overline{X} + \overline{Y}\) for the entire statement, and subsequently applying \((X + Y)^{\overline{}} = \overline{X} \cdot \overline{Y}\) to each premise, we obtain the expression:
\[
L \cdot \overline{G} \cdot \overline{N} + G \cdot \overline{J} + G \cdot \overline{P} + O \cdot \overline{J} + P \cdot J + N \cdot \overline{(P + L) \cdot O} + \overline{L} + \overline{O} + J \tag{1}
\]
This equation (1) is subsequently subjected to the simplification \(J + O \cdot \overline{J} = J + O\), leading to a tautological outcome, as \(\overline{O} + O = 1\). This tautology is consistent across Scenarios 3, 4, 7, and 8; regardless of the conditions that \(H_1: (L \to (G + N))\) or \((L \to G \cdot N)\), and whether \(H_2\) with \((G \to J)\) or without \((G \to J)\).

When \((M = L) \implies J\), Equation (1) transforms into:
\[
L \cdot \overline{G} \cdot \overline{N} + G \cdot \overline{J} + G \cdot \overline{P} + O + P \cdot J + N \cdot \overline{(P + L) \cdot O} + \overline{L} + J \tag{2}
\]
Applying \(X + X \cdot Y = X\) yields \(P \cdot J + J = J\), consequently simplifying the expression (2) to:
\[
L \cdot \overline{G} \cdot \overline{N} + G \cdot \overline{J} + G \cdot \overline{P} + O + N \cdot \overline{(P + L) \cdot O} + \overline{L} + J \tag{3}
\]
Further application of \(X + \overline{X} \cdot Y = X + Y\) results in:
\[
N \cdot \overline{(P + L) \cdot O} + O = N \cdot \overline{(P + L)} + N \cdot \overline{O} + O = N \cdot \overline{(P + L)} + N + O = N + O
\]
Thus, the simplified expression (3) becomes:
\[
L \cdot \overline{G} \cdot \overline{N} + G \cdot \overline{J} + G \cdot \overline{P} + O + N + \overline{L} + J \tag{4}
\]
\[
= \overline{L} + G \cdot \overline{N} \cdot \overline{N} + G \cdot \overline{J} + G \cdot \overline{P} + O + N + J
\]
\[
= \overline{L} + N + \overline{G} + G \cdot \overline{J} + G \cdot \overline{P} + O + J
\]
\[
= \overline{L} + N + \overline{G} + \overline{J} + \overline{P} + O + J
\]
The expression \(\overline{J} + J = 1\) leads to a tautological outcome, as shown in Scenario 2. In scenarios where \((G \to J)\) is absent, the expression (4) simplifies to
\[
L \cdot \overline{N} \cdot G \cdot P \cdot \overline{O} \to J,
\]
as shown in Scenario 1. When replacing \((L \to (G + N))\) with \((L \to G \cdot N)\), the expression becomes:
\[
L \cdot \overline{(G \cdot N)} + G \cdot \overline{J} + G \cdot \overline{P} + O + N + \overline{L} + J
\]
\[
= L \cdot (\overline{G} + \overline{N}) + G \cdot \overline{J} + G \cdot \overline{P} + O + N + \overline{L} + J
\]
\[
= \overline{L} + L \cdot (\overline{G} + \overline{N}) + N + G \cdot \overline{J} + G \cdot \overline{P} + O + J
\]
\[
= \overline{L} + \overline{G} + \overline{N} + N + G \cdot \overline{J} + G \cdot \overline{P} + \overline{L} + J \tag{5}
\]
The expression (5) leads to a tautological outcome because \(\overline{N} + N = 1\) regardless of \(H_2\) with \((G \to J)\) or without \((G \to J)\), as shown in Scenarios 5 and 6.
